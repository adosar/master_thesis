\documentclass{article}

\usepackage[T1]{fontenc}
\usepackage{ebgaramond}
\usepackage[dvipsnames]{xcolor}
\usepackage{lipsum}
\usepackage{chemmacros}

\usepackage[english, greek]{babel}

\title{
	\color{Blue}\Large\bfseries
	ΑΠΟ ΤΗ ΔΥΝΑΜΙΚΗ ΕΝΕΡΓΕΙΑΚΗ ΕΠΙΦΑΝΕΙΑ
	\\[0.2cm]
	\large ΣΤΗΝ
	\\[0.1cm]
	\Large ΠΡΟΣΡΟΦΗΣΗ ΑΕΡΙΩΝ
	\\[0.2cm]
	\large ΜΕΣΩ
	\\[0.1cm]
	\Large ΒΑΘΙΑΣ ΜΑΘΗΣΗΣ
	\vspace{1.5cm}
}

\author{
	{\scshape\Large Αντωνιος Π. Σαρικας}
	\\[0.5cm]
	Επιβλέπων Καθηγητής: \scshape{Γεωργιος Ε. Φρουδακης}
}

\date{\scshape Ηρακλειο, 2024\vspace{1cm}}

\newcommand{\MOFs}{\foreignlanguage{english}{MOFs}}
\newcommand{\threed}{\foreignlanguage{english}{3D}}
\begin{document}

\maketitle
\pagenumbering{gobble}

Τα μεταλο-οργανικά πλέγματα, ή εν συντομία \MOFs{}, λόγω των μεγάλων επιφανειών
τους θεωρούνται πολλά υποσχόμενα υλικά για εφαρμογές που περιλαμβάνουν
προσρόφηση αεριών. Ωστόσο, η συνδυαστική φύση τους μεταφράζεται σε έναν πρακτικά
άπειρο χώρο υλικών, καθιστώντας επίπονη την ταυτοποίηση καινοτόμων υλικών με
παραδοσιακές τεχνικές. Τα τελευταία χρόνια, προσεγγίσεις μηχανικής μάθησης έχουν
αναπτυχθεί, επιτρέποντας στους ερευνητές να φιλτράρουν σε πολύ μικρό χρόνο
τεράστιες βάσεις δεδομένων από \MOFs{} με τη βοήθεια προβλεπτικών μοντέλων. Η
ποιότητα των τελευταίων εξαρτάται σε μεγάλο βαθμό από την μαθηματική
αναπαράσταση του υλικόυ, με αποτέλεσμα να είναι αναγκαία η χρήση εισόδων που
περιγράφουν κατάλληλα το υλικό. Σε αυτήν την εργασία, προτείνουμε ένα
γενικευμένο σχήμα για την πρόβλεψη ιδιοτήτων προσρόφησης, χρησιμοποιώντας ως
μόνη είσοδο τη δυναμική ενεργειακή επιφάνεα. Θεωρούμε την τελευταία ως μια
\threed{} ενεργειακή εικόνα και την περνάμε μέσα από ένα \threed{} συνελικτικό
νευρωνικό δίκτυο, γνωστό για την ικανότητα του να επεξεργάζεται εικόνες. Η
προτεινόμενη μέθοδος εφαρμόζεται σε \MOFs{} για την πρόβλεψη προσρόφησης
διοξειδίου του άνθρακα. Το μοντέλο που προκύπτει ξεπερνά σε απόδοση ένα
συμβατικό μοντέλο που χρησιμοποιεί μόνο γεωμετρικά χαρακτηριστικά. Επιπρόσθετα,
δείχνεται ότι η μέθοδος μπορεί να εφαρμοστεί σε διαφορετικά συστήματα
αερίων-υλικών, εξετάζοντας την προσρόφηση μεθανίου σε ομοιοπολικά οργανικά
πλέγματα. Η ακρίβεια και ο γενικός χαρακτήρας της προτεινόμενης μεθόδου, την
καθιστούν κατάλληλη για φιλτράρισμα μεγάλης κλίμακας.

\end{document}
