\begin{abstract}
	\lettrine[
	nindent=0em, findent=0.5em, loversize=-0.12, lines=5
	]{\initfamily{M}}{\bfseries\color{Blue}etal-organic frameworks}, or in short
	\acrshortpl{mof}, thanks to their ultra high porosity and surface area, are
	deemed as prominent candidates for applications involving gas adsorption.
	However, their intrinsic combinatorial nature translates to a practically
	infinite material space, rendering the identification of novel materials
	with traditional methods cumbersome. Over the last years, \acrlong{ml}
	approaches based on predictive models have been developed, allowing
	researchers to rapidly screen large databases of \acrshortpl{mof}. The
	quality of these models is highly dependent on the mathematical representation of
	a material, thus necessitating the use of informative inputs. In this
	thesis, we propose a generalized framework for predicting gas adsorption
	properties, using as one and only input the \acrlong{pes}. We treat the
	latter as a 3D energy image and then pass it through 3D \acrlong{cnn}, known
	for its ability to process image-like data. The proposed pipeline
	is applied in \acrshortpl{mof} for predicting \ch{CO2} uptake. The resulting
	model outperforms both in terms of accuracy and data efficiency a
	conventional one built upon textual properties. Additionally, we demonstrate
	the transferability of the approach to other host-guest systems, by
	examining \ch{CH4} uptake in \acrlongpl{cof}. The performance and generality
	of the proposed approach along with the fast input calculation thanks to
	parallelization, render it suitable for large scale screening. Finally,
	discussion for improving and extending the suggested scheme is provided.
\end{abstract}
