\chapter{Introduction}

\lettrine[
	nindent=0em, findent=0.5em, loversize=-0.12, lines=5
]{\initfamily{R}}{\bfseries\color{Blue}eticular chemistry}\index{Reticular
chemistry}\index{Reticular chemistry}, a field that bridges inorganic and
organic chemistry \parencite{Yaghi2020}, has emerged from a simple albeit
powerful idea: \emph{combining molecular building blocks to form extended
crystalline structures} \parencite{yaghi}. It all started in 1990s, with the
advent of \glspl{mof}\index{Metal-organic frameworks}, the first ``offspring''
of reticular chemistry. \glspl{mof}, a class of nanoporous materials
\emph{composed of metal ions\index{Metal ions} or clusters\index{Metal clusters}
coordinated to organic ligands\index{Organic ligands} aka organic
linkers\index{Organic linkers}}, possess extraordinary properties, such as
ultrahigh porosity and huge surface areas \parencite{Farha_2012}. To get a sense
of how extraordinary these materials are, it is suffice to say that \emph{one
gram of such a material can have a surface area as large as a soccer field}.
The fact that reticular materials are ``brought to life'' by combining
simple building blocks, allows chemists and material scientists to design materials
in a judicious manner. The epitome of design in reticular chemistry is found in
the synthesis of a zirconium-based \gls{mof} \parencite{Alezi2016},
incorporating the polybenzene network or ``cubic graphite'' structure, predicted
about 70 years ago.

\section{Applications of Reticular Chemistry}

Owing to their aforedescribed properties along with their extremely tunable and
modular nature, \glspl{mof} have been considered prominent solutions for
gas-adsorption related problems \parencite{Li2007, Jiang2022}.  \glspl{mof} find
application in fields such as gas storage\index{Gas storage} and
separation\index{Gas separation}, catalysis\index{Catalysis} and drug
delivery\index{Drug delivery}, just to name a few.

\emph{\bfseries Carbon capture}\index{Carbon capture} is a prime example
\parencite{An_2009, Sumida2011, Qazvini2021}, where \gls{mof}-based sorbents
have been deemed as green, low-cost and energy-efficient solutions. These
materials provide versatile solutions to carbon capture, spanning various phases
of the capture process, with \gls{dac}\index{Direct air capture} being a
noteworthy example.  \gls{dac} includes chemical or physical methods for
extracting carbon dioxide directly from the ambient air, with MOF-powered
\gls{dac} showing great potential as a green and sustainable strategy for
reducing carbon dioxide levels, contributing to the combating of climate change
\parencite{Bose2023}.

\emph{\bfseries Hydrogen storage}\index{Hydrogen storage} is one of the greatest
challenges of hydrogen economy, currently inhibiting the transition from fossil
fuels to hydrogen. Fortunately, characteristics of \gls{mof} adsorbents such as
fast adsorption/desorption kinetics, low operating pressures and high hydrogen
capacities, render them as promising answers to the aforementioned challenge
\parencite{Suh2011, Suresh_2021}.

Methane is an attractive fuel for vehicular applications, being a relatively
clean-burning fuel compared to gasoline. \emph{\bfseries Methane
storage}\index{Methane storage} in sorbents known as \gls{ang} exhibit
advantages over \gls{cng} and \gls{lng}, both in terms of energy-efficiency and
vehicular safety. \glspl{mof} \parencite{Ma2007, Spanopoulos_2016,
Tsangarakis2023} and their ``reticular siblings'' \glspl{cof}\index{Covalent
organic frameworks}---composed only of light elements---show great promise as
\gls{ang} solutions \parencite{MendozaCortes2011, Furukawa_2009, Martin2014,
Tong2018}.

\section{The Problem}

The intrinsic combinatorial character of reticular chemistry, translates to
practically an infinite number of realizable structures. Currently, the
\gls{csd} contains more than \num{100000} experimentally synthesized \glspl{mof}
\parencite{siegel29} while the arrival of in silico designed \glspl{mof}
\parencite{siegel36, Rosen2021, Chung2019, chong51, DeVos2023, trillions,
Boyd_2019} has immensely expanded the available material pool. The huge size of
current and future \gls{mof} databases\index{Database} \parencite{trillions} is
both a blessing and a curse for the identification of novel materials. Blessing,
since a large number of candidate structures doesn't limit our choices and as
such, the chances to find the right material for a given problem. Curse, since
the enormous size of \glspl{mof} space makes it harder for researchers to
efficiently explore it, complicating the tracing of materials with the desired
properties. It is therefore crucial to find a way that allows us to efficiently
explore such a huge material space. Another way to rephrase our problem is the
following: \emph{\bfseries Given a large catalog of \glspl{mof}, is there a way
to efficiently filter out the most promising ones for the application of
interest?}

As a first approach to deal with this challenge, one could, in principle
experimentally synthesize and characterize each one of the materials listed in
the given catalog. Although \emph{experimental synthesis and
characterization}\index{Experimental synthesis}\index{Experimental
characterization} is the ultimate way to assess the performance of a
material\footnote{As Richard Feynmann said: \emph{``The test of all knowledge is
experiment. Experiment is the sole judge of scientific truth''}.}, the fact that
a single laboratory study can take days or even months, renders experimental
techniques impractical.  A more efficient approach is computational
screening\index{Computational screening} based on \emph{molecular
simulations}\index{Molecular simulations}, which for years has served as the
principal tool for the discovery of high-performing \glspl{mof}
\parencite{chong55, chong56, chong57, chong58, chong59}. Although computational
screening dramatically accelerates the assessment of a single material compared
to experimental techniques, brute-force screening of current and upcoming
databases is considered suboptimal, given the size of the latter.

\Gls{ml}\index{Machine learning} aka \emph{data-driven\index{Data-driven}
techniques} come to the rescue when dealing with \emph{big data}\index{Big data}
and over the last years have picked up the torch from molecular simulations
regarding material characterization. Given a collection of
\emph{input-output}\index{Input}\index{Output} pairs, i.e. a mathematical
representation of a material and a corresponding property, a \gls{ml}
algorithm\index{Machine learning algorithm}\footnote{Note that \gls{ml}
algorithms are not limited to solve only such kind of problems, which fall under
the umbrella of supervised learning\index{Supervised learning}. See subsection
\ref{subsec:paradigms} for other types of problem tackled by \gls{ml}.} seeks to
\emph{uncover the underlying structure-property relationship}. To put it in a
nutshell, a \gls{ml} algorithm ``eats'' \emph{data}\index{Data}\footnote{The
data may come either from experiments, simulations or a combination of the two.}
and ``spits out'' a \emph{model}\index{Model}, which can be \emph{used to sort a
large catalog of MOFs in just few seconds}. Obviously, for \gls{ml} approaches
to be effective and reliable, it is necessary that the resulting models are of
high quality.

\begin{itemize}
	\item Decide if figures must be added
	\item Add missing citations
\end{itemize}

%\begin{figure}
%	\centering
%	\begin{tikzpicture}
%		\begin{axis}[
%				axis background/.style={fill=blue!10},
%				axis lines=box, width=0.5\textwidth,
%				title=Material space,
%				xlabel=$\leftarrow$ Metal cluster $\rightarrow$,
%				ylabel=$\leftarrow$ Organic linker $\rightarrow$,
%				xlabel near ticks, ylabel near ticks,
%				domain=-4:4, view={0}{90}, colormap/viridis, ticks=none,
%			]
%			\addplot3 [contour filled={number=20}]
%				{pow(x^2 + y - 11, 2) + pow(x + y^2 - 7, 2)};
%			\addplot[
%				only marks,
%				mark options={fill=white, mark=pentagon*, scale=1.5pt,}
%			] coordinates {
%				(-3.5, 3.5) (-3.5, -3)
%				(-3, 3) (-2.2, 3)
%				(2.5, 2) (3, -2) (3.5, 0)
%			};
%		\end{axis}
%	\end{tikzpicture}
%	\caption{ADD CAPTION}
%	\label{fig:mofs}
%\end{figure}

\section{Literature Review}

\emph{One of, if not the most important factor for the performance of \gls{ml}
models, is the way we select to mathematically represent materials or
molecules}. In other words, the type and amount of chemical information that is
``injected'' into these representations commonly known as
\emph{descriptors}\index{Descriptor}, can make the difference between a
high-performing and a baseline model. As such, it is of uttermost importance to
employ descriptors that provide sufficient information for the properties of
materials or molecules we are interested in to predict.

With regards to gas adsorption\index{Gas adsorption} in \glspl{mof}, one of the
first and most commonly used descriptors, are the so called
\emph{geometric}\index{Geometric descriptors} ones, which capture the pore
environment of the framework. This type of descriptors includes textual
characteristics of \glspl{mof} such as void fraction\index{Void fraction},
gravimetric surface area\index{Gravimetric surface area} and pore limiting
diameter\index{Pore limiting diameter}. Although \gls{ml} models build with
these descriptors work particularly well at the high pressure regime
\parencite{Fernandez2013, Wu2020, Dureckova_2019}, their performance
deteriorates when adsorption takes place at low pressures or the guest molecules
exhibit non-negligible electrostatic interactions with the framework atoms. This
performance drop should be expected, since geometric descriptors completely
ignore the ``cornerstone'' of adsorption: \emph{host-guest
interactions}\index{Host-guest interactions}.

In order to improve the performance of \gls{ml} models and bypass the
limitations of geometric descriptors in the aforementioned conditions, another
type of descriptors known as \emph{energy-based} descriptors \index{Energy-based
descriptors} \parencite{chong127, Shi_2023, robust, Orhan2023}, has been
introduced. This type of descriptors supply \gls{ml} algorithms with information
regarding the energetics of adsorption, and can be used standalone or in
combination with geometric descriptors.

In one of the first works to fingerprint the energetic landscape of \glspl{mof}
\parencite{bucior}, energy histograms\index{Energy histogram} derived from the
interactions of guest molecules with the framework atoms were used to predict
hydrogen and methane uptake with remarkable accuracy. Prior to calculating the
energy histograms, a three dimensional grid is overlayed on the unit cell of the
\gls{mof}. Next, at each point of the grid, the interaction between the guest
molecule with the framework atoms is calculated, producing a three dimensional
energy grid\index{Energy grid}. Finally, this energy grid is converted into a
histogram, by partitioning the energy values of the grid into bins of specific
energy width. By using solely these histograms as descriptors---without
including any textual property---Bucior and his coworkers trained Lasso
regression models\index{Regression}\index{Lasso}, for predicting:
\begin{enumerate*}[label=\roman*).]
	\item \ch{H2} swing capacity\index{Swing capacity} between \SI{100}{\bar}
		and \SI{2}{\bar} at \SI{77}{\kelvin}
	\item \ch{CH4} swing capacity\index{Swing capacity} between \SI{65}{\bar}
		and \SI{5.8}{\bar} at \SI{298}{\kelvin}
\end{enumerate*}.
The resulting models were extremely accurate, achieving a \gls{mae}\index{Mean
absolute error} of \SI{2.3}{\gram\per\liter} and
\SI{12.9}{\cubic\centi\meter\per\cubic\centi\meter} for \ch{H2} and \ch{CH4},
respectively, tested on the hMOFs database \parencite{siegel36}.

\section{Objective of Thesis}
